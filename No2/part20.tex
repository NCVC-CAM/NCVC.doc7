%!TEX root = ../NCVC7.tex
\mysection{スキャニングパスのNCデータを生成}

\subsection{IGESデータの読み込み}
 図~\ref{fig:sample.iges} のIGESデータをNCVCで読み込みます.
この時点で \ref{sec:AboutIGES}~節にも書いたようにNCVCが落ちる場合があります.
IGESタイプを変更するなど適宜対応してください.

\begin{figure}[H]
\centering
\includegraphics[scale=0.5]{No2/fig/fig21.png}
\caption{サンプル図形の読み込み}
\label{fig:ncvc21}
\end{figure}

\subsection{スキャニングパスの生成}
 スキャニングパスを生成するには,切削対象となるNURBS曲面とガイドとなるNURBS曲線を1つずつ選択する必要があります.
マウスの左クリックで選択してください.選択順は問いませんが,マウスが動くとドラッグ扱いになるので慎重にクリックしてください.
選択できると選択色
\footnote{\menu{オプション>表示属性>表示属性の設定}から\menu{共通}タブの\menu{選択オブジェクト}の色}
に変わります.

\begin{figure}[H]
\centering
\includegraphics[scale=0.5]{No2/fig/fig22.png}
\caption{NURBS曲面とNURBS曲線の選択}
\label{fig:ncvc22}
\end{figure}

 NURBS曲面とNURBS曲線が選択されていると,\menu{ファイル>NCデータの生成>スキャニングパスの生成}(\keys{F2})のメニューが有効になります.
図~\ref{fig:ncvc23} のダイアログから適当な値を設定してください.

\begin{itemize}
\item [曲面からのオフセット]は削り代と考えてください.
\textbf{ここをゼロにすると曲面に沿ってスキャニングされるので,実質的な仕上げ工程になります.}
\item [NC生成時ワーク上面をZ軸のゼロにする]にチェックが入っていると[ワークの高さ]で設定した高さがZ軸のゼロになるようにNCデータが生成されます.
ワークの上面でZ軸の原点や工具長補正を設定する場合がほとんどだと思うので,通常はチェックを入れておきましょう.
\end{itemize}

\begin{figure}[H]
\centering
\includegraphics{No2/fig/fig23.png}
\caption{スキャニングパスの設定}
\label{fig:ncvc23}
\end{figure}

 図~\ref{fig:ncvc23} で \keys{OK} を押すと,図~\ref{fig:ncvc24} のようにスキャニングパスが表示されます.
選択されたガイド曲線が[スキャニングラインの分割数]で分割され,さらにそのガイド曲線に沿うように点群が生成されます.

 データによっては計算がうまくいかずデータが欠落する場合があります.その場合は選択したガイド曲線を変えてみてください.
図~\ref{fig:ncvc22} の選択方法では左上付近のパスが欠けてしまいました.

\begin{figure}[H]
\centering
\includegraphics[scale=0.5]{No2/fig/fig24.png}
\caption{スキャニングパスの表示Ⅰ}
\label{fig:ncvc24}
\end{figure}

 図~\ref{fig:ncvc22} ではモデルの手前(または奥)にあるX軸と平行なガイド曲線を選択したので,点群は 図~\ref{fig:ncvc24} のようにY方向の集まりになりますが,
モデルの右(または左)にあるY軸と平行なガイド曲線を選択すると,図~\ref{fig:ncvc25} のように点群はX方向の集まりになります.
\textbf{どちら方向に切削するかは,このガイド曲線の選択によって変わります.}

\begin{figure}[H]
\centering
\includegraphics[scale=0.5]{No2/fig/fig25.png}
\caption{スキャニングパスの表示Ⅱ}
\label{fig:ncvc25}
\end{figure}

\subsection{NCデータの出力}
 \menu{ファイル>NCデータの生成>3D切削データの出力}(\keys{F7})のメニューから出力できます.
出力ファイル名にはデフォルトで \_Scan というサフィックス(接尾語)が付けられます(図~\ref{fig:ncvc26}).

 切削条件は2D切削と共通ですが,図~\ref{fig:ncvc26} から \keys{編集}ボタンを押すと,図~\ref{fig:ncvc27} のように NURBS Mode となり,不要なタブは非表示にされます.
切削条件ファイルの拡張子(.nci)が共通というだけなので,3D切削用に設定しなおし,それ用に新規で保存しておきましょう.

\begin{minipage}{0.5\textwidth}
\begin{figure}[H]
\centering
\includegraphics[scale=0.7]{No2/fig/fig26.png}
\caption{NCデータの出力設定}
\label{fig:ncvc26}
\end{figure}
\end{minipage}
\begin{minipage}{0.5\textwidth}
\begin{figure}[H]
\centering
\includegraphics[scale=0.7]{No2/fig/fig27.png}
\caption{NURBS用の切削条件設定}
\label{fig:ncvc27}
\end{figure}
\end{minipage}

\vspace*{2zh}
 カスタムヘッダーも3D切削用に書き換えてください.
G92での原点指示は小型加工機に多い設定方法ですが,
ワーク座標系と工具長補正で設定を行う大型加工機の場合は,リスト~\ref{lst:header.txt} の方が一般的かもしれません.
ほか工具交換のGコード挿入など,積極的にカスタマイズしてください.

\begin{lstlisting}[caption=カスタムヘッダーの例,numbers=none,label=lst:header.txt]
%
G90G54G00X0Y0  → G92は削除  G54ワーク座標系のXY原点に移動
{Spindle}M03
G43Z50.H01     → 工具長補正でZ50.0まで移動
\end{lstlisting}

\subsection{NCデータのシミュレーション結果}
 図~\ref{fig:ncvc23} で指定したボールエンドミル半径が,NCVC用のOpenGL表示コメント

\vspace*{0.5zh}
\begin{breakbox}
\vspace*{-0.25cm}
\small
\begin{alltt}
(Endmill=R_)
\end{alltt}
\vspace*{-0.1cm}
\end{breakbox}

としてカスタムヘッダーの "\%" の次の行に挿入されます.
最大矩形はNCデータから自動算出されるので,図~\ref{fig:ncvc29} のように,ほぼ正確な切削イメージがシミュレーションできます.

 実はこのIGESデータでスキャニングパスを生成すると,\ref{sec:KodatunoMsg}~節の警告メッセージが出力されます.
そのせいで 図~\ref{fig:ncvc29} をよく見ると中央付近にパス漏れと思われる若干の盛り上がりが確認できます.

\begin{figure}[H]
\centering
\includegraphics[scale=0.5]{No2/fig/fig28.png}
\caption{スキャニングパスNCデータのシミュレーション結果Ⅰ}
\label{fig:ncvc28}
\end{figure}

\begin{figure}[H]
\centering
\includegraphics[scale=0.5]{No2/fig/fig29.png}
\caption{スキャニングパスNCデータのシミュレーション結果Ⅱ}
\label{fig:ncvc29}
\end{figure}

\newpage
 もし期待通りのシミュレーション結果ではない場合,

\begin{itemize}
\item \keys{F6}(ウィンドウの切り替え)でIGESデータのウィンドウにする.
\item \keys{F2}(スキャニングパスの生成)で 図~\ref{fig:ncvc23} から再設定(選択状態が維持されているはず).
\item \keys{F7}(3D切削データの出力)でNCデータを出力.
\end{itemize}

のようにショートカットキーを駆使して効率よく作業を進めることが可能です.
