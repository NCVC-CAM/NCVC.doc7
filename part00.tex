%!TEX root = ./NCVC7.tex
\section*{はじめに}
 この機能のコア部分は,金沢大学マンマシン研究室で開発されているKodatunoライブラリを用いて実現しています.
\begin{center}
\url{http://www-mm.hm.t.kanazawa-u.ac.jp/research/kodatuno/}
\end{center}
 このページでも述べられているとおり,まだ完全なライブラリではありません(僕が言うのも変ですがww).
IGESデータの読み込みに失敗してもイライラしてはいけません.
望み通りのパスが出ないからといってSNSに悪口を書いてはいけません.
生あたたかく見守ってください.

 3次元切削においては,2次元切削以上に注意が必要です.
生成されたNCデータは,工具干渉なども含めた検証を十分に行った上で,加工作業を進めてください.
\textbf{トラブルや損害等について,作者は一切責任を持ちません.必ず自己責任で使用してください.}

\vspace*{2zh}
 Kodatunoライブラリの使用に際し,同研究室の関係各位には大変お世話になりました.
ここに謝意を表します.
